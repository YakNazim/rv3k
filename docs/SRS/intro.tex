\subsection{Purpose}
  The purpose of this document is to document the
  requirements for the RocketView 3000 (RV3K) \vers software system. This
  document is a reference for software scope and development. It will explain
  RV3K’s purpose, features, interfaces and system constraints, as well as the 
  scope of use.

  Functional requirements were elicited during conversations between February 7th,
  2017, and \today{} with Jamey Sharp and Andrew Greenberg at Portland State
  University. \cite{Interview}

  This document is (very loosely) based on IEEE SA - 29148-2011 Software
  Requirements Specification (SRS) guidelines \cite{ieee}.

\subsection{Scope}
  RV3K \vers is a PSAS rocket telemetry display module that allows local and
  remote users to monitor PSAS vehicle flights using a networked computer 
  that supports modern web-browsers (2017). The goal is to create data 
  visualization features that will allow users to better understand what is
  presented. This includes, but is not limited to, 3D maps and video.

\subsection{Product overview}

\subsubsection{System purpose}
  The three main purposes of RV3K are:
  \begin{itemize}
    \item State visualization and state transparency
      \begin{itemize}
        \item[--] To display internal vehicle and mission state information
        \item To display external vehicle and mission state information
      \end{itemize}
    \item Publicity
    \begin{description}
      \item To display telemetry data and mission state to viewers who are no domain experts
    \end{description}
    \item Logistics
      \begin{itemize}
        \item To coordinate recovery
          \begin{itemize}
            \item To locate individual ground crew
            \item To locate rocket parts for recovery
          \end{itemize}
      \end{itemize}
  \end{itemize}


\subsubsection{Product perspective}
  \begin{itemize}
      \item RV3K is a cross-platform, software intensive system with a browser-based user interface
      \item The RV3K user interface is composed of web browser modules and
        widgets, server modules, and internetworking capabilities.
        As such, RV3K requires hardware that supports modern
        browsers (see \S 2.1.2).
  \end{itemize}

\subsubsection{Objectives and success criteria}
  \begin{itemize}
    \item System can be invoked cross-platform via command
    \item RV3K UI can be invoked from the web browsers listed in \S 2.1.2
    \item RV3K correctly performs calculations on incoming data
    \item Upon failure, RV3K will enforce halting over displaying incorrect data 
    \item RV3K displays telemetry information provided in the data stream
    \item RV3K displays GPS coordinates of the recovery crew
  \end{itemize}

\subsubsection{Limitations}
  RV3K, nor its creators shall be held responsible for death or injury resulting
  directly or indirectly from its use. Users of RV3K assume full responsibility
  for accidents, damages, injury, death, expenses, and/or humiliation incurred
  while using RV3K software system. This limitation is covered by the Capstone
  contract statement: ``No mission critical or phase based projects''

  \url{http://wiki.cs.pdx.edu/capstone/sponsors.html}

  \medskip

  \subparagraph{Scientific accuracy}
    The accuracy of calculations for vehicle location, attitude, trajectory, and
    any other internal and external state are limited by sensor states, signal strength,
    connectivity, and other factors of complexity. RV3K shall provide approximate telemetry data
    about these states, but it shall not guarantee scientific accuracy or precision.



\subsection{Definitions}
\begin{description}
  \item[cross-platform:] An application that runs on Linux, macOS and Windows operating systems (2017).
\end{description}

\subsubsection{Abbreviations used}

\begin{description}
  \item[RV3K] Rocket View 3000 \vers Telemetry Viewer
  \item[APRS] Automatic Packet Reporting System
  \item[PSAS] Portland State Aerospace Society
  \item[UI]   User Interface
  \item[URL]  Uniform Resource Locator
\end{description}
